\title{Integrated Parametric Shaft Generator}

%Add/Remove authors as you please
%Numbers in [] correspond to affiliations below
%\emails must be before \affils

\author[1]{Mohamed Khalil}
\author[1]{Zeno Korondi}
\emails{mohamed.khalil@tum.de, korondi.zeno@gmail.com}
\author[2]{Mohamed Elhaddad}
\author[2]{Tino Bog}
\emails{mohamed.elhaddad@tum.de, tino.bog@tum.de}

\affil[1]{Master's Computational Mechanics students, Technical University in Munich}
\affil[2]{Chair of Computation in Engineering, Technical University in Munich}

\maketitle

\begin{abstract}
The Integrated Parametric Shaft Generator (IPSG) is an integrated framework for designing and analysis of shafts, a vital component in a plethora of mechanical applications. This framework allows the user to flexibly build the shaft's geometry and mount common shaft features such as fillets, chamfers and keyways, as well as apply the necessary loads at given sections and conduct a structural finite elements analysis, all under one graphical user interface (GUI). The analysis is being conducting using the finite cell method (FCM) for higher order FE analysis, developed at the Chair of Computation in Engineering at Technical University in Munich. This work has also involved implementing a polar cell grid to conform to the cylindrical base geometry of the shafts, aiming to reduce the complexity of the FCM analysis using cartesian grid. The efficiency of both methods have been compared and presented in this work.
\end{abstract}
