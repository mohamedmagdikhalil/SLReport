\subsection{Data Structure}
% How the program was structured, how the data was stored, transmitted between the python interface and Adhoc++, boost wrapping of the analysis kernel. A UML diagram would be good here.

The IPSG is composed of three major blocks:
\begin{itemize}
\item{Shaft Data}: This block is scripted in Python language and is concerned with the data storage of the shaft details as well as its components (loads, features, etc.)
\item{Analysis Kernel}: This block is scripted in C++ as part of Adhoc++ and is concerned with conducting FCM analysis on the shaft, obtaining its geometry from STL files generated by OpenCascade libraries utilized by the GUI.
\item{GUI}:
\tododone[inline]{Mohamed}{responsible for \textit{GUI}: Zeno}
\end{itemize}

The link between the first two blocks is indicated in the UML class diagram in figure ~\ref{uml}. The interface \emph{AnalysisKernelFcmWrapper} was responsible for wrapping the functionalities of the c++ class \emph{AnalysisKernelFcm} to make them accessible through the corresponding Python class \emph{ShaftAnalysisKernel} using \emph{boost::python} library.

\begin{figure}
  \begin{center}
    \includegraphics[width=\textwidth]{./images/uml.png}
    \caption{UML Diagram for Data Structure of the Shaft and Analysis Kernel}
    \label{fig:uml}
  \end{center}
\end{figure}